\title{A Statistical Analysis of Chicago Traffic}
\author{Tristan Rasmussen}
\date{\today}

\documentclass[12pt]{article}
\usepackage{multicol}
\usepackage{amsmath}
\usepackage{amsfonts}
\usepackage{bm}
\usepackage{graphicx}
\usepackage{float}
\usepackage[caption = false]{subfig}
\usepackage{hyperref}
  
\begin{document}
\maketitle
\begin{multicols*}{2}
\section{Introduction}
Everybody hates traffic. Some studies have show that traffic ends up costing Americans over 120 billion dollars a year \cite{costoftraffic}. However, before trying to address the problem of traffic it is imperative that we form accurate statistical models of traffic flow and the variables that affect it. In this paper, we perform both a parametric and nonparametric analysis of traffic in the city of Chicago.
\section{Description of the Data}
For the purposes of this paper we will be examining Chicago traffic data conjunction with data about sports games in the city. These data sets were obtained separately and then combine together.
\subsection{Chicago Congestion}
The main data set of interest is a record of estimated traffic speeds in 29 regions of Chicago gathered from 03/12/2011 until 03/26/2013 \cite{regiondataset}. For the purposes of this data set the City of Chicago is partitioned into 29 regions, and each data point is the speed for a specific region at a given date and time. Speeds were calculated by sending GPS probes to CTA buses that were operating in the specific region at the given time. Measurements were taken every 10 minutes and a record was kept of the speed, in addition to the number of buses in the region and the number of valid GPS reads that were used when calculating the speed.

In addition to data about regions, the City of Chicago provides more detailed data about 1250 individual segments of road \cite{segmentdataset}. Some preliminary analysis was done of this data set, but it had a much more restricted time frame (2013-1-15 - 2013-2-28) and so analysis was eventually abandoned. However, we will use this data set to try to explain some of the phenomenon observed in the region data.
\subsection{Chicago Sports Games}
The second data set of interest was a collection of all sports games that have happened in Chicago during the period of interest (03/12/2011 - 03/26/2013). We predicted that sports games would have a significant impact on traffic and so would be a useful feature to add to the data set. Unfortunately, we were unable to find a single comprehensive list of all sports games in Chicago and were forced to go to individual sites. We gathered the dates of every Bulls \cite{bullsdata}, Bears \cite{bearsdata}, Whitesox \cite{whitesoxdata}, Cubs \cite{cubsdata} and Blackhawks \cite{blackhawksdata} game in the time period and created a data set where each point was a date and the games that occurred on that date.
\subsection{Data Preprocessing}
Once the above datasets were gathered they were combine into a single data set that contained all of the information we wanted. Each timestamp was broken down into year, month, day, hour, minute, day of week, and a flag indicating if it was a weekend or not. In addition a day label was added that indexed the days sequentially. Finally we added a column that was true if there was any sports game on a day and false otherwise. Once all of this preliminary processing was done, we cleaned up the data set by removing any data points that had invalid speeds (either caused by having 0 bus reads or by having a speed over 40 mph when most roads have a speed limit of 30 mph). 
\subsection{Preliminary exploration and Visualization}
In order to provide tools for visualizing and exploring the data and the models that we have constructed, we have created a web application (hosted [here]) that allows for an interactive exploration of the data overlaid on a map of Chicago.

[Some additional plots here]
\subsection{Goals}
The ultimate goal of this paper is to build a statistical model for traffic in the city of Chicago. There are two components that we will explore in this paper: estimating the probability distribution of traffic in Chicago and building a regression model for the data. The goal of building a density estimate is [what?]. The goal of building a regression model is to provide predictive power for the average traffic speed based on the parameters given above.
\subsection{Methods}
When performing regression analysis, we have withheld about 25\% of the data to use as a test set. This data has been selected randomly from the entire training set. The model is trained on the remaining 75\% and then tested against the test set. Individual models have been built for each region.
\section{Parametric Analysis}
\begin{table}
\begin{tabular}{|l|l|l|}
\hline
Region ID & $\mu$ & $\sigma$\\\hline
1 & 24.697801 $\pm$ 0.022228 & 2.931518\\\hline
2 & 28.991722 $\pm$ 0.019415 & 2.614829\\\hline
3 & 24.967138 $\pm$ 0.024994 & 3.448271\\\hline
4 & 23.052651 $\pm$ 0.025400 & 3.429280\\\hline
5 & 25.890058 $\pm$ 0.023019 & 3.058450\\\hline
6 & 24.246442 $\pm$ 0.024921 & 3.393697\\\hline
7 & 23.805981 $\pm$ 0.023463 & 3.155498\\\hline
8 & 22.587990 $\pm$ 0.025805 & 3.578877\\\hline
9 & 25.737242 $\pm$ 0.022743 & 2.969804\\\hline
10 & 24.507356 $\pm$ 0.021532 & 2.999743\\\hline
11 & 24.226650 $\pm$ 0.026715 & 3.716619\\\hline
12 & 20.780952 $\pm$ 0.031236 & 4.281174\\\hline
13 & 19.058748 $\pm$ 0.031712 & 4.419897\\\hline
14 & 26.446459 $\pm$ 0.024094 & 3.272564\\\hline
15 & 27.356264 $\pm$ 0.024014 & 3.307650\\\hline
16 & 28.148639 $\pm$ 0.027105 & 3.469245\\\hline
17 & 30.015234 $\pm$ 0.024633 & 3.053233\\\hline
18 & 27.013029 $\pm$ 0.023081 & 3.201800\\\hline
19 & 26.337392 $\pm$ 0.020474 & 2.850558\\\hline
20 & 27.081739 $\pm$ 0.022002 & 3.003730\\\hline
21 & 26.525628 $\pm$ 0.023895 & 3.221976\\\hline
22 & 30.250759 $\pm$ 0.019234 & 2.580613\\\hline
23 & 24.408021 $\pm$ 0.020131 & 2.806418\\\hline
24 & 26.036382 $\pm$ 0.022289 & 3.019421\\\hline
25 & 31.352320 $\pm$ 0.020673 & 2.620213\\\hline
26 & 29.973643 $\pm$ 0.022522 & 3.063198\\\hline
27 & 30.945219 $\pm$ 0.021275 & 2.845948\\\hline
28 & 33.359715 $\pm$ 0.036218 & 3.719515\\\hline
29 & 23.721448 $\pm$ 0.030303 & 3.865192\\\hline
\end{tabular}
\end{table}
\subsection{Density Estimation}
As outlined above, we are interested in modelling the distribution of traffic speeds in the city of Chicago by region. To perform a parametric estimate of the density we must first decide on a model that we believe the speeds follow, and then we can calculate maximum likelihood parameters to produce a viable model.

To begin with we assumed that traffic speed followed a normal distribution. [Why would we believe this?]. Using this model it is relatively easy to calculate the maximum likelihood values for the mean and standard deviation. Computing this over the 29 regions gives the data found in columns 1 and 2 of Table \ref{•}.

In addition to simply computing the distribution, we can examine how well this distribution fits the actual data graphically and numerically. To examine the applicability of our model graphically we created QQ plots. Several of these plots look relatively close to linear, indicating that the underlying distributions are close to normal. However, there are also several plots which exhibit strong non linearity; this indicates that the normal distribution model may not be entirely accurate for all of the regions in question.

%In an attempt at a parametric estimate of the density we first made the assumption that the data was normally distributed. Under this assumption we could fit a density by simply calculating the sample mean and standard deviation and using these to create density estimates. 

%To asses the accuracy of our normality assumptions we produced QQ plots for each region. A cursory inspection of each individual region seems to show that the normality assumption is not a very accurate one for this data set. [Insert some images here]. We quantify this later with a KS test.
%[I need some justification for this]
\subsection{Regression Analysis}
The second goal we have outlined above is to produce a viable model of traffic using time and sports games as parameters. We chose to take two different parametric approaches to this problem: a simple linear regression, and a higher order cubic regression.
\subsubsection{Linear Regression}
Linear regressions are the gold standard for an initial shot at predicting a model from input. This model assumes that the data is a linear function of the independent variables.  In our analysis we trained several different linear models and chose the one that had the lowest testing error of any linear model. We then used this model to build a fit for the entire data set.


Having looked at the density of traffic speeds we then turned our investigation to modelling traffic speed as a function of time and sports games. We tried two different classes of regressions for each region: a simple linear model and a polynomial model. A number of different models were tried and the one with the lowest test error was ultimately chosen as the winning model. We present the MSE on the testing data produced by both the linear model and the polynomial model, along with the winning model. We report below the MSE produced by the best model on the entire training data set, the $r^2$, and the model that produced this fit.
\section{Nonparametric Analysis}
\subsection{Density Estimation}
To begin with we wanted to asses the accuracy of the normal density created above, but we were unable to go beyond a simple QQ plot. Once we enter the realm of nonparametric statistics we can quantify this by performing a KS test against the predicted value. The following table gives the predicted KS values.

Having seen the failings of the normal density estimate, we can attempt to create a better density estimate using a kernel density estimate. These were created and we produced 95\% confidence bands.
\subsection{Regression Analysis}
To try to improve upon the regressions produced by the parametric models, we can try to produce a local linear regression. These have been produced and we looked at some stuff.
\section{Discussion}
\subsection{Density Estimation}
In this case it seems that the assuming a normal distribution was quite a mistake. 
\subsection{Regression Analysis}
Nonlinear does pretty well, and the local linear regression does ok as well. And some other conclusions.
\section{Biblography}
\begin{thebibliography}{9}

\bibitem{costoftraffic}
  \url{http://www.cebr.com/reports/the-future-economic-and-environmental-costs-of-gridlock/}

\bibitem{regiondataset}
  \url{https://data.cityofchicago.org/Transportation/Chicago-Traffic-Tracker-Historical-Congestion-Esti/emtn-qqdi}
  
\bibitem{segmentdataset}
  \url{https://data.cityofchicago.org/Transportation/Chicago-Traffic-Tracker-Historical-Congestion-Esti/77hq-huss}

\bibitem{bullsdata}
  \url{http://www.landofbasketball.com/results/2010_2011_scores_full.htm}

\bibitem{cubsdata}
  \url{http://chicago.cubs.mlb.com/schedule/sortable.jsp?c_id=chc&year=2011}

\bibitem{bearsdata}
  \url{http://www.repole.com/sun4cast/data.html#dataprior}
	
\bibitem{whitesoxdata}
  \url{http://www.baseball-almanac.com/teamstats/schedule.php?y=2013&t=CHA}

\bibitem{blackhawksdata}
  \url{http://www.hockey-reference.com/teams/CHI/2011_games.html}

\end{thebibliography}
\end{multicols*}
\end{document}